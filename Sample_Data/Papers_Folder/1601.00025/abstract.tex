People typically learn through exposure to visual concepts associated with linguistic descriptions. For instance, teaching visual object categories to children is often accompanied by descriptions in text or speech. In a machine learning context, these observations motivates us to ask whether this learning process could be computationally modeled to learn visual classifiers\ignore{ recognition accuracy, compared to traditional   approaches that depend only on the visual domain}. More specifically, the main question of this work is how to utilize purely textual description of visual classes\ignore{, as privileged information,} with no\ignore{or few} training images, to learn explicit visual classifiers for them. We propose and investigate two baseline formulations, based on regression and domain transfer that predict a linear classifier. Then, we propose a new constrained optimization formulation that combines a regression function and a knowledge transfer function with additional constraints to predict the linear classifier parameters for new classes. We also propose a generic kernelized  models where a kernel classifier, in the form defined by the representer  theorem, is predicted. The kernelized models allow defining any two RKHS kernel functions in the visual space and text space, respectively, and could be useful for other applications. We finally propose a kernel function between unstructured text descriptions that builds on distributional semantics, which shows an advantage in our  setting  and could be useful for other applications. We applied all the studied models to predict visual classifiers for two fine-grained categorization datasets, and the results indicate successful predictions of our final model against  several baselines that we designed.  


%Our approach is based on learning two submodels. The first submodel is a regression function from the text features to the visual classifier space. The second submodel is a correlation function between the textual features and the visual features. Having learnt these two submodel, based on the information of the seen classes, our approach predicts the classifier of the unseen class by solving a quadratic program that involves the regression and the correlation function. The  proposed approach was comprehensively evaluated on Birds and Flower Datasets, and the results indicate successful predictions of our model against  the baseline that we designed. 